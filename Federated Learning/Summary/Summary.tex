%! TEX TS-program = xelatex
\documentclass[12pt, a4paper]{article}

\usepackage{ctex}
\usepackage[margin=0.5in]{geometry}
\usepackage{cite}

\title{
    联邦学习总结
}
\author{
    曲俊全 戴立森
}
\date{
    \today
}

\begin{document}
    \maketitle
    \section*{削减通信代价的主要算法}
        \subsection*{Communication-Efficient Learning of Deep Networks from Decentralized Data\cite{mcmahan2017communicationefficient}, 2017}
            FedAvg算法,在一次全局的聚合计算中能够加入更多的本地参数更新,但是计算代价增加了,在非IID用户中效果不好。
        \subsection*{A Communication Efficient Collaborative Learning Framework for Distributed Features\cite{liu2020communication}}
            和FedAvg算法类似,保证了在纵向联邦学习中的收敛。缺点是计算代价增加,如果全局聚合不频繁的话收敛会延迟。
        \subsection*{Two-stream instabilities in a plasma\cite{1965AuJPh..18..271K}}
            受迁移学习启发,提出的双流模型,在联邦学习中,固定全局模型,加速了模型的收敛速度。但是,计算代价增加,收敛会出现滞后现象。
        \subsection*{Two-stream instabilities in a plasma\cite{1965AuJPh..18..271K}}
            受迁移学习启发,提出的双流模型,在联邦学习中,固定全局模型,加速了模型的收敛速度。但是,计算代价增加,收敛会出现滞后现象。

\end{document}