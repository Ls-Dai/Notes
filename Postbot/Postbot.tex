%! TEX program = xelatex

\documentclass[10pt]{beamer}

\usepackage[heading = true]{ctex}
% \usepackage[colorlinks,linkcolor=red]{hyperref}

\usepackage{graphicx}
\usepackage{float}
\usepackage{ulem}

\usepackage{subfiles}

\usetheme[]{Berkeley}

\mode<presentation>

\title{
    {自动发帖机器人调研}
}
\author{戴立森}
\date{Nov 10, 2020}

\begin{document}
    \maketitle

        \begin{frame}
            \frametitle{信息生成}
            \begin{itemize}
                \item[-]{
                    发帖操作
                    \begin{enumerate}
                        \item 利用社交媒体给出的接口,或者直接在网页上模拟人的交互,进行发帖操作。
                        \item 利用社交媒体给出的接口,或者利用爬虫等抓取方式,获取网络上的信息以供后端处理。
                    \end{enumerate}
                    \hspace*{\fill}\\
                    }
                \item[-]{
                    生成帖子
                    \begin{enumerate}
                        \item 利用已知\textbf{信息}生成新的信息
                        \item 自动识别并且直接借用他人素材、链接,和现有的其他信息搭配
                    \end{enumerate}
                    } 
            \end{itemize}
        \end{frame}

        \begin{frame}
            \frametitle{信息生成}
            \begin{itemize}
                \item[-]{
                    发帖操作
                    \begin{enumerate}
                        \item 利用社交媒体给出的接口,或者直接在网页上模拟人的交互,进行发帖操作。
                        \item 利用社交媒体给出的接口,或者利用爬虫等抓取方式,获取网络上的信息以供后端处理。
                    \end{enumerate}
                    \hspace*{\fill}\\
                    }
                \item[-]{
                    生成帖子
                    \begin{enumerate}
                        \item 利用已知\textbf{信息}生成新的信息
                        \item 自动识别并且直接借用他人素材、链接,和现有的其他信息搭配
                    \end{enumerate}
                    } 
            \end{itemize}
        \end{frame}
        
\end{document}
