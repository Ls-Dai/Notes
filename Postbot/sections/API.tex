%! TEX program = xelatex

\documentclass[../Postbot.tex]{subfiles}


\begin{document}

	\begin{frame}
		\frametitle{\href{https://developer.twitter.com/en/products/twitter-api}{推特API}}
		\begin{columns}
			\begin{column}{.2\textwidth}
				\centering
				{\normalsize 推文}
				\hspace*{\fill}
				\hspace*{\fill}
			\end{column}

			\begin{column}{.8\textwidth}
				\begin{itemize}
					\item 发布、接受、使用推文
					\item 推特时间轴
					\item 整理推文集
					\item 搜索推文:七天内的
					\item 按照需求过滤实时推文
					\item 样例实时推文
				\end{itemize}
			\end{column}

			\begin{column}{.8\textwidth}
				\begin{itemize}
					\item 发布、接受、使用推文
					\item 推特时间轴
					\item 整理推文集
					\item 搜索推文:七天内的
					\item 按照需求过滤实时推文
					\item 样例实时推文
				\end{itemize}
			\end{column}
			
		\end{columns}


	\end{frame}

	\begin{frame}
		\frametitle{信息生成}
		\begin{itemize}
			\item[-]{
				发帖操作
				\begin{enumerate}
					\item 利用社交媒体给出的接口,或者直接在网页上模拟人的交互,进行发帖操作。
					\item 利用社交媒体给出的接口,或者利用爬虫等抓取方式,获取网络上的信息以供后端处理。
				\end{enumerate}
				\hspace*{\fill}\\
				}
			\item[-]{
				生成帖子
				\begin{enumerate}
					\item 利用已知\textbf{信息}生成新的信息
					\item 自动识别并且直接借用他人素材、链接,和现有的其他信息搭配
				\end{enumerate}
				} 
		\end{itemize}
	\end{frame}

\end{document}