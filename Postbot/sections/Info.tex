%! TEX program = xelatex

\documentclass[../Postbot.tex]{subfiles}

\begin{document}
    \begin{frame}
        \frametitle{信息生成}
        通过实际应用层面的需求,可以看出,大的方面来说,一个“自动发帖机器人”需要有如下功能: \\
        \begin{itemize}
            \item 建立“用户画像”,针对自己的“兴趣、职业”,进行发帖等操作;
            \item 建立“用户关注画像”,针对自己的“关注话题”,进行转帖等操作;
        \end{itemize}
    \end{frame}

    \begin{frame}
        \frametitle{发帖信息生成}
        \centering
        根据“用户的兴趣和特点”建立初始的“用户画像”,并且,我们可以建立一个初始数据库,由此:\\
        \hspace*{\fill}\\
        \hspace*{\fill}\\
        \begin{columns}
            \begin{column}{.3\textwidth}
                \centering
                \begin{enumerate}
                    \item 文字(内容或是话题)
                    \item 图片
                    \item 视频
                \end{enumerate}
            \end{column}
            \begin{column}{.7\textwidth}
                \begin{enumerate}
                    \item 文字引用生成评论-NLP特征提取、NLP生成模型; \\
                    \item 图片配文-图片特征提取、NLP生成模型; \\
                    \item 视频配文-图片、视频特征提取,NLP生成模型; \\
                    \item 经由图片的图片生成-GAN图像翻译; \\
                    \item 经由文字的文字生成-NLP生成模型; \\
                    \item 为自己的关注用户、关注话题和关注内容生成分类标签 \\
                    \item @ 符合内容特征的关注用户 \\
                \end{enumerate}
            \end{column}
        \end{columns}
    \end{frame}

    \begin{frame}
        \frametitle{用户画像完善}
        \centering
        在初始“用户画像”上,根据“关注点”和“(地区)热点”而收到信息:\\
        \hspace*{\fill}\\
        \hspace*{\fill}\\
        \begin{columns}
            \begin{column}{.3\textwidth}
                \centering
                接收到:
                \begin{enumerate}
                    \item 文字
                    \item 图片
                    \item 视频
                    \item 整体贴子
                \end{enumerate}
            \end{column}
            \begin{column}{.7\textwidth}
                NLP特征提取、NLP生成模型、随机森林等等
                \begin{enumerate}
                    \item 根据文字、图片、视频的特征,是否加入数据库 \\
                    \item 根据整体帖子的所有特征(内容、关注人群、热度、地区)综合,是否转帖 \\
                    \item 根据整体帖子的所有特征(内容、关注人群、热度、地区)综合,是否进入关注组 \\
                    \item 根据整体帖子的反馈,扩大“关注点” \\
                    \item 根据其他用户和自己的“关注点”修改自己的用户画像 \\
                    \item ... \\
                \end{enumerate}
            \end{column}
        \end{columns}
    \end{frame}

    \begin{frame}
        数据池说明 \\
        \begin{itemize}
            \item 内容库(文字(话题)、图片等)
            \item 关注点库
            \item 用户画像库(自己和其他)
            \item ...
        \end{itemize}
    \end{frame}

\end{document}