%! TEX program = xelatex

\documentclass[../Postbot.tex]{subfiles}

\begin{document}
    
    \begin{frame}
        \frametitle{任务结构}
        \begin{itemize}
            \item[-]{
                API 接口设置
                \begin{enumerate}
                    \item 利用社交媒体给出的接口,或者直接在网页上模拟人的交互,进行发帖操作。
                    \item 利用社交媒体给出的接口,或者利用爬虫等抓取方式,获取网络上的信息以供后端处理。
                \end{enumerate}
                \hspace*{\fill}\\
                }
            \item[-]{
                生成帖子
                \begin{enumerate}
                    \item 利用已知\dotuline{信息}生成新的信息
                    \item 自动识别并且直接借用他人素材、链接,和现有的其他信息搭配
                \end{enumerate}
                } 
        \end{itemize}
    \end{frame}


    \begin{frame}
        \frametitle{API 接口简述}
        \begin{itemize}
            \item[-]{
                读取
                \begin{enumerate}
                    \item 调用开发者提供的接口读取
                    \item 爬取网页内容
                \end{enumerate}
                \hspace*{\fill}\\
                }
            \item[-]{
                写入
                \begin{enumerate}
                    \item 调用开发者提供的接口读取
                    \item 爬取网页前置接口,分析,并且记录特征,调用之
                \end{enumerate}
                } 
        \end{itemize}
    \end{frame}

    \begin{frame}
        \frametitle{信息生成}
        \begin{itemize}
            \item[-]{
                生成独立信息
                \qquad 不依赖其他信息,仅仅依靠简单的信息(比如话题)创生出内容来。可以是一段文字,一个其他话题,一个图片,甚至是一个视频。\\
                \hspace*{\fill} \\
                }
            \item[-]{
                生成依赖信息
                \qquad 依赖其他信息,生成相关的“不抄袭”的信息,多见于模仿创作,或者文章、图片和视频“解读”式创作。 \\
                \hspace*{\fill} \\
                } 
            \item[-]{
                生成关联信息
                \qquad 关联其他帖子,生成相关的“不抄袭”的信息,多见于评论,或者图片视频二次创作。 \\
                \hspace*{\fill} \\
                } 
        \end{itemize}
    \end{frame}

\end{document}