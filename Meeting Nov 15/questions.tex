%!TEX program = xelatex

\documentclass[10pt]{article}

\usepackage{ctex}
\usepackage[colorlinks, linkcolor=red, urlcolor=blue]{hyperref}

\title{
    问题回答稿
}
\author{戴立森}
\date{Nov 14, 2020}

\begin{document}
    \maketitle
    \begin{itemize}
        \item [{\large Q1}]{
            {\large 简单介绍一下您在UCB选择的课程} \\
            通过和学院的教学副院长和有关老师的沟通,结合学校的实际要求和个人兴趣,我选择了一门计算机课程,一门数学课和一门电路课程。我的专业是电子信息,热门课程有助于更好地了解前沿知识,以及世界上的同龄同行的人究竟在做什么。
        }
        \item [{\large Q2}]{
            {\large 为什么坚定地选择去美国留学/参加美国的交流项目} \\
            最开始选择去交流,最大的愿望是能够开拓眼界,在美国的学术文化和科技氛围中积淀自身素养。限制于本科生的认知和任务,对于科技创新的理解和工作并不能很深入,因为本科生接触教学是要大于科研的。而如今研究生阶段做出这种选择,就是在此基础上,进一步参与到前沿科创中,
            进一步借助这种环境提升自己的能力。未来成为一个对社会发展有所贡献的个体。
            }
        \item [{\large Q3}]{
            {\large 在UCB/UCI有没有印象深刻的教授/同学,发生了什么事情} \\
            有一门计算机课的美国教授非常酷,风趣而且幽默。他喜欢让我们去他的Office Hour,大家一起思考一个编程问题,最后让先做出来的人给出答案并点评,考试结束后和每个人握手,告诉我们你们都完成了这门课,很棒。 \\
            此外,做实验的时间里,有一个聪明的同学提前做完了,然后就在实验室的黑板上画漫画吐槽这门课,引来每个做完的人都在上面“涂鸦”…… \\
            很多事情都让我印象深刻,无论是学习和生活,文化上的差异确实十分有趣。
            }
        \item [{\large Q4}]{
            {\large 大工作为985高校,提供给你了哪些更优质的资源和便利的平台?} \\
            如果要从“硬实力”来说,大工作为一所工科院校,科研实力和教学水平是广泛为海外大学的工学院所认可的。很多海外机构都愿意和大工合作,而大工也给了我们机会,给了愿意去接受不同学习环境的同学各种各样的机会来拓宽国际化视野。 \\
            而从“软实力”的方面来说,我也看见大工的本科生教学和科创也积极寻求变革,开放、包容、创新、启发,不仅是要与世界接轨,也要走在世界的前列。这一点会极大地帮助我们日后的学习生活。
            }
        \item [{\large Q5}]{
            {\large 选择交流项目的依据和获取信息的渠道?} \\
            大工为我们提供了各个地区的丰富的交流项目。选择对应的交流项目时,一要考虑好自己的目标,二要结合自己的兴趣。目标可以大到具体利用对方学校的哪些资源,也可以小到仅仅是对自己思维的重构;兴趣可以是一个专业的前沿,也可以是一个教学文化的体验。 \\
            大工国际处的网站为我们提供了汇总了相关信息,而学部学院的网站上也有特定项目的通知,丰富而多彩。如果你有特别感兴趣的学校,也可以去官网查查他们的交流项目,确认大工认可后可以提交申请。
            }
        \item [{\large Q6}]{
            {\large 您参加大工交换项目有什么收获和体验?} \\
            实践了前沿知识、拓宽了视野,也锻炼了思维方式和综合能力。
            }
        \item [{\large Q7}]{
            {\large 您觉得国内国外教学模式及课程安排有何异同?} \\
            大的方面来说国内和国外教学的差异是不大的。理论、考试、实践的组合也大同小异。一门课和另一门课不同,关键还是在于老师和同学,和国别的区分不太大。UCB、UCI是十分优秀的学校,我们有很多需要向它们学习。
            }
    \end{itemize}
    \centering
    {\small \href{https://github.com/Ls-Dai/Notes/blob/main/Meeting%20Nov%2015/questions.tex}{文档源代码}}
\end{document}