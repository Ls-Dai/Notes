%!TEX program = xelatex

\documentclass[12pt]{article}
\usepackage[heading = true]{ctex}
\usepackage[colorlinks, linkedcolor=red]{hyperref}
\usepackage{graphicx}

\title{
    {云服务器研讨会策划书}
}

\date{Nov 10, 2020}
\author{联邦学习项目组}

\begin{document}
    \maketitle

    \section*{时间和地点}
        \begin{center}
            {\large Nov 12, 2020, 星期四} \\
            {\normalsize 7:00 P.M.} \\ 
            % 上一行,时间需要调整,另外这个地方要写上具体的时间,时分。
            % 这个时间需要我们做调查敲定,确保来一定的人
            \hspace*{\fill} \\
            {\large 创新园大厦,16楼人工智能学院,1601} \\
            {\small 或者,\href{https://meeting.tencent.com/s/5A5VOzWXucmi}{Online Tencent Meeting}} \\
            % 线上线下同步进行,上面要写会议号等信息
            % 郭老师推荐晚上

        \end{center}
        
    \section*{主讲人}
        \begin{center}
            戴立森
        \end{center}

    \section*{参加者(排名不分先后)}
        \begin{center}
            郭艳卿\,老师\ 王波\,老师\ 付海燕\,老师\ 刘航\,老师\ 李
            \includegraphics[scale=0.15]{scr/img/yi.jpg}
            老师\\
            实验室研究生\quad 等 
            % 上面要添加如下人:四位老师
            % 还有这些老师的所有学生
            % 他们应该选线上或者线下的其中一种方式
        \end{center}

    \clearpage

    \section*{讨论主题}
        \subsection*{1. 云服务器概况}
            云服务器服务介绍、配置、费用等说明。
        \subsection*{2. 云服务器使用}
            现场演示云服务器的开机使用和关闭等操作,包括各种远程交互。
        \subsection*{3. 加速并行计算}
            简要说明并行计算的原理、讲解并行计算的核心代码。
        \subsection*{4. 云服务器使用规则}
            云服务器的管理方式和使用规范。

    \section*{相关材料}
    \href{https://github.com/Ls-Dai/Cloud-Sever-Tutorial}{云服务器使用说明和相关规定}

\end{document}